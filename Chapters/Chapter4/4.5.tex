\section*{4.5}
\begin{enumerate}
    \item Show how the Intermediate Value Theorem follows as a corollary to Theorem 4.5.2.
    
    \begin{proof}
    Let \( f:[a,b] \rightarrow \mathbb{R} \) be continuous and \( L \) a real number between \( f(a) \) and \( f(b) \). Since \( [a,b] \) is connected and \( f \) is continuous, by Theorem 4.5.2 we have that \( f([a,b]) \) is connected. Since every connected set in \( \mathbb{R} \) is an interval it follows that \( f([a,b]) \) is also an interval. Since \( f(a), f(b) \in f([a,b]) \) and \( f([a,b]) \) is an interval, it follows that every element between \( f(a) \) and \( f(b) \) is in \( f([a,b]) \). In particular \( L \in f([a,b]) \). Thus there \( c \in (a,b) \) such that \( f(c) = L \).  
    \end{proof}
    
    \item Decide on the validity of the following conjectures.
    \begin{enumerate}
        \item Continuous functions take bounded open intervals to bounded open intervals.
        \item Continuous functions take bounded open intervals to open sets.
        \item Continuous functions take bounded closed intervals to bounded closed intervals.
    \end{enumerate}
    \begin{proof}
    \begin{enumerate}
        \item This is false. \( f(x) = x^{2} \) is continuous and takes \( (-1,1) \) to \( [0,1) \).
        \item This is false. See the above.
        \item This is true. Notice a bounded closed interval is compact. By Theorem 4.5.2 we know that that the image, by a continuous function, of a bounded closed interval must also be an interval. However, from Theorem 4.4.2 it must also be compact which implies that it must be bounded and closed. Therefore, it must be a bounded and closed interval. 
    \end{enumerate}
    \end{proof}
    
    \item Is there a continuous function on all of \( \mathbb{R} \) with range \( f(\mathbb{R}) \) equal to \( \mathbb{Q} \)?
    \begin{proof}
    No. If \( f \) is continuous on \( \mathbb{R} \) and \( f(\mathbb{R}) = \mathbb{Q} \), then there exist \( a,b \in \mathbb{R} \) such that  \( f(a) = 0 \) and \( f(b) =1 \). Suppose \( a < b \). Then \( f([a,b]) \subset \mathbb{Q} \) must be connected. However, we know that there exists \( y \in \mathbb{R} \setminus \mathbb{Q} \) such that \( 0 < y < 1 \). Thus, there must exist \( c \in (a,b) \) such that \( f(c) = y \) contradicting with \( f(\mathbb{R}) = \mathbb{Q} \). A similar argument demonstrates the same holds when \( b < a \).
    \end{proof}
    
    \item A function \( f \) is increasing on \( A \) if \( f(x) \leq f(y) \) for all \( x < y \) in \( A \). Show that the Intermediate Value Theorem does have a converse if we assume \( f \) is increasing on \( [a,b] \).
    
    \begin{proof}
    Let \( f \) have the IVP and be increasing on \( [a,b] \) and let \( \epsilon > 0 \) be given.
    \begin{enumerate}
    
    \item \underline{Continuity at \( a \)}: Let \( y \in (a,b] \). If
    \[
    f(y) < f(a) + \epsilon
    \]
    then, by \( f \) increasing, \( a < z < y \) implies
    \[
    f(a) \leq f(z) \leq f(y) < f(a) + \epsilon
    \]
    so that
    \[
    \vert f(z) - f(a) \vert < \epsilon
    \]
    On the other hand, if \( f(y) \geq f(a) + \epsilon \) then, by IVP, there exists \( c \in (a,y) \) such that \( f(c) = f(a) + \frac{\epsilon}{2} < f(a) + \epsilon \). Then, by our previous argument, we will have that \( a < z < c \) implies that \( \vert f(z) - f(a) \vert < \epsilon \). Therefore, \( f \) is continuous at \( a \). 
    
    \item \underline{Continuity at \( b \)}: Extremely similar to the above.
    
    \item \underline{Continuity on \( (a,b) \)}: Let \( c \in (a,b) \). Notice \( f \) is then increasing and has the IVP on \( [a,c] \). So following from (b), we will get that there exists \( \delta_{1} > 0 \) such that for all \( x \) such that \( x < c \) and \( c-x < \delta_{1} \) we will have \( \vert f(x) - f(c) \vert < \epsilon \). Similarly \( f \) is increasing and has the IVP on \( [c,b] \). So following form (a), we will get that there exists \( \delta_{2} > 0 \) such that for all \( x \) such that \( c < x \) and \( x-c < \delta_{2} \) we will have \( \vert f(x) - f(c) \vert < \epsilon \). Thus if \( \delta = \min\{\delta_{1},\delta_{2}\} \) then \( \vert x - c \vert < \delta \) implies that \( \vert f(x) - f(c) \vert < \epsilon \). Thus \( f \) is continuous at \( c \).
    \end{enumerate}
    Therefore, altogether, we have that \( f \) is continuous on \( [a,b] \).
    \end{proof}
    
    \item Finish the proof of the Intermediate Value Theorem using the Axiom of Completeness started previously.
    \begin{proof}
    Continuing the argument, we first show that \( f(c) \leq 0 \). We observe that either \( c \in K \) or, by \( c \) being the supremum of K, there is \( (x_{n}) \subset K \) such that \( x_{n} \rightarrow c \). The former implies that \( f(c) \leq 0 \). On the other hand, in the latter case, by compactness of \( [a,b] \), it follows then that \( c \in [a,b] \) and so \( f(c) \) is well-defined. By continuity of \( f \), we get \( f(x_{n}) \rightarrow f(c) \). Now, since \( f(x_{n}) \leq 0 \) for all \( n \in \mathbb{N} \), it follows from the Order Limit Theorem that \( f(c) \leq 0 \). Thus, in all cases \( f(c) \leq 0 \). Now, since \( f(b) > 0 \), it follows that \( c \neq b \). Furthermore for all \( x \in (c,b] \) we have \( f(x) > 0 \). If \( (y_{n}) \subset (c,b] \) such that \( y_{n} \rightarrow c \), then, again, by continuity of \( f \), we must have \( f(y_{n}) \rightarrow f(c) \). This then implies that \( f(c) = 0 \), for otherwise we would have that \( f(y_{n}) > 0 \) for all \( n \) and \( \lim_{n \rightarrow \infty} f(y_{n}) = f(c) < 0 \) contradicting with the Order Limit Theorem. Therefore, \( f(c) = 0 \). Now then, if \( g: [a,b] \rightarrow \mathbb{R} \) is continuous on \( [a,b] \) and \( L \in (g(a),g(b)) \) then, by the previous argument, there exists \( c \) such that \( g(c) - L = 0 \). Thus there exists \( c \) such that \( g(c) = L \). This also holds when \( g(b) < g(a) \).
    \end{proof}
    
    \item Finish the proof of the Intermediate Value Theorem using the Nested Interval Property started previously.
    \begin{proof}
    Continuing the argument, we denote each of the intervals by \( I_{n} = [x_{n},y_{n}] \). By the NIP, we must have that there exists \( c \in \cap_{n=0}^{\infty} I_{n}\). Furthermore, \( x_{n},y_{n} \rightarrow c \). By continuity of \( f \) we have then that \( \lim_{n \rightarrow \infty} f(x_{n}) = \lim_{n \rightarrow \infty} f(y_{n}) = f(c) \).  Since \( f(x_{n}) < 0 \) for all \( n \) and \( f(y_{n}) \geq 0 \) for all \( n \), it follows by the Order Limit Theorem, that \( f(c) \geq 0 \) and \( f(c) \leq 0 \). Therefore \( f(c) = 0 \). Thus, if \( g: [a,b] \rightarrow \mathbb{R} \) is continuous on \( [a,b] \) and \( L \in (g(a),g(b)) \) it follows from the previous argument that there exists \( c \) such that \( g(c) - L = 0 \). Therefore, there exists \( c \in (a,b) \) such that \( g(c) = L \). The same holds when \( g(b) < g(a) \). 
    \end{proof}
    
    \item Let \( f \) be a continuous function on the closed interval \( [0,1] \) with range also contained in \( [0,1] \). Prove that \( f \) must have a fixed point; that is, show \( f(x) = x \) for at least one value of \( x \in [0,1] \).
    
    \begin{proof}
    Notice that \( f(0) \geq 0 \) and \( f(1) \leq 1 \). Define \( g(x) = x - f(x) \). Clearly \( g(0) = 0 - f(0) \leq 0 \) and \( g(1) = 1 - f(1) \geq 0 \). Moreover, by the Algebraic Continuity Theorem, we know that \( g \) is continuous on \( [0,1] \). Thus by the Intermediate Value Theorem, there exists \( c \in (0,1) \) such that \( g(c) = 0 \). That is, there is \( c \in (0,1) \) such that \( c - f(c) = 0 \), which, in turn, implies that \( c = f(c) \). Therefore, \( f \) has a fixed point in \( [0,1] \).
    \end{proof}
    
    
    
    
    
    
    
    
    
    
    
    
    
    
    
    
    
    
    
    
    
    
    
    
    
    
    
    
    
    
\end{enumerate}