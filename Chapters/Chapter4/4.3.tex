\section*{4.3}
\begin{enumerate}
    \item Let \( g(x) = \sqrt[3]{x} \).
    \begin{enumerate}
        \item Prove that \( g \) is continuous at \( c = 0 \).
        \item Prove that \( g \) is continuous at a point \( c \neq 0 \). (The identity \( a^{3}-b^{3} = (a-b)(a^{2}+ab+b^{2}) \) will be helpful.)
    \end{enumerate}
    \begin{proof}
    \begin{enumerate}
        \item We first observe that
        \[
        \vert \sqrt[3]{x} \vert = \sqrt[3]{\vert x \vert}
        \]
        and that if \( x_{1} \leq x_{2} \) then \( \sqrt[3]{x_{1}} \leq \sqrt[3]{x_{2}} \), that is say, \( \sqrt[3]{x} \) is increasing. The first observation is obvious. To demonstrate the latter observation suppose, to the contrary, that \( \sqrt[3]{x_{2}} < \sqrt[3]{x_{1}} \). Then we would have
        \[
        x_{2} = \sqrt[3]{x_{2}}^{3} < \sqrt[3]{x_{1}}^{3} = x_{1}
        \]
        which contradicts with our assumption that \( x_{1} \leq x_{2} \). So, if \( \epsilon > 0 \) is given, then choosing \( \vert x \vert < \epsilon^{3} \) it follows that
        \[
        \vert \sqrt[3]{x} \vert = \sqrt[3]{\vert x \vert} < \sqrt[3]{\epsilon^{3}} = \epsilon
        \]
        Therefore, \( \sqrt[3]{x} \) is continuous at \( 0 \).
        
        \item Let \( c \neq 0 \) and let \( \epsilon > 0 \) be given. We observe that
        \[
        x- c = (x^{\frac{1}{3}}-c^{\frac{1}{3}})(x^{\frac{2}{3}}+x^{\frac{1}{3}}c^{\frac{1}{3}}+c^{\frac{2}{3}})
        \]
        If we let \( y = x^{\frac{1}{3}} \) then we get that
        \[
        x^{\frac{2}{3}}+c^{\frac{1}{3}}x^{\frac{1}{3}}+c^{\frac{2}{3}} = y^{2}+c^{\frac{1}{3}}y+c^{\frac{2}{3}}
        \]
        which means that this expression is quadratic in \( y \). Furthermore, one can easily verify, using the quadratic formula, that since \( c \neq 0 \) it follows that \( c \) is not a root of \( y^{2}+c^{\frac{1}{3}}y+c^{\frac{2}{3}} \). Thus, since \( x^{\frac{1}{3}} \) is continuous at \( 0 \) it follows that there exists \( \delta > 0 \) and \( M_{2} > M_{1} > 0 \) such that whenever \( \vert x - c \vert < \delta_{1} \) we will have that
        \[
        M_{1} < \vert x^{\frac{2}{3}}+c^{\frac{1}{3}}x^{\frac{1}{3}}+c^{\frac{2}{3}} \vert < M_{2}
        \]
        Notice this implies that \( \frac{M_{1}}{\vert x^{\frac{2}{3}}+c^{\frac{1}{3}}x^{\frac{1}{3}}+c^{\frac{2}{3}} \vert} < 1 \) for \( x \) close enough to \( c \). So, if \( \delta = \min \{ d_{1}, \epsilon M_{1} \} \) then it follows that whenever \( \vert x-c \vert < \delta \) we get
        \[
        \vert x^{\frac{1}{3}} - c^{\frac{1}{3}} \vert = \frac{\vert x-c \vert}{ \vert x^{\frac{2}{3}}+c^{\frac{1}{3}}x^{\frac{1}{3}}+c^{\frac{2}{3}} \vert } < \frac{\epsilon M_{1}}{\vert x^{\frac{2}{3}}+c^{\frac{1}{3}}x^{\frac{1}{3}}+c^{\frac{2}{3}} \vert} < \epsilon
        \]
        which was to be demonstrated.
    \end{enumerate}
    \end{proof}
    
    \item \begin{enumerate}
        \item Supply a proof for Theorem 4.3.9 using the \( \epsilon \)-\(\delta \) characterization of continuity.
        
        \item Give another proof of this theorem using the sequential characterization of continuity.
    \end{enumerate}
    
    \begin{proof}
    \begin{enumerate}
        \item Let \( \epsilon > 0 \) be given. By continuity of \( g \) at \( f(c) \) we know that there exists \( \delta_{1} > 0 \) such that \( \vert f(x) - f(c) \vert < \delta_{1} \) implies \linebreak \( \vert g(f(x)) - g(f(c)) \vert < \epsilon \). Furthermore, by continuity of \( f \) at \( c \), we know that there exists \( \delta_{2} > 0 \) such that \( \vert x - c \vert < \delta_{2} \) implies \linebreak \( \vert f(x) - f(c) \vert < \delta_{1} \). Thus, for \( \vert x - c \vert < \delta_{2} \) we have that 
        \[
        \vert g(f(x)) - g(f(c)) \vert < \epsilon
        \]
        as desired.
        
        \item Let \( (x_{n}) \rightarrow c \) with \( x_{n} \in A \). Since \( f \) is continuous at \( c \), it follows \( f(x_{n}) \rightarrow f(c) \). But then, by continuity of \( g \) at \( f(c) \) it follows \( g(f(x_{n})) \rightarrow g(f(c)) \). 
    \end{enumerate}
    \end{proof}
    
    \item Using the \( \epsilon \)-\( \delta \) characterization of continuity (and thus using no previous results about sequences), show that the linear function \( f(x) = ax+b \) is continuous at every point of \( \mathbb{R} \). 
    \begin{proof}
    Let \( \epsilon > 0 \) be given and let \( c \in \mathbb{R} \). If \( \vert x - c \vert < \frac{\epsilon}{\vert a \vert} \) then it follows that
    \[
    \vert f(x) - f(c) \vert = \vert ax+b -ac-b \vert = \vert ax - ac \vert = \vert a \vert \vert x - c \vert < \vert a \vert \frac{\epsilon}{\vert a \vert} = \epsilon
    \]
    Thus \( ax+b \) is continuous on \( \mathbb{R} \). 
    \end{proof}
    
    \item \begin{enumerate}
    \item Show using Definition 4.3.1 that any function \( f \) with domain \( \mathbb{Z} \) will necessarily be continuous at every point in its domain.
    
    \item Show in general that if \( c \) is an isolated point of \( A \subseteq \mathbb{R} \), then \( f:A \rightarrow \mathbb{R} \) is continuous at \(c \).
    \end{enumerate}
    
    \begin{proof}
    \begin{enumerate}
        \item Let \( \epsilon > 0 \) be given. Notice \( \vert x-c \vert < \frac{1}{2} \) and \( x \in \mathbb{Z} \) implies \( x=c \). Thus for all such \( x \)
        \[
        \vert f(x) - f(c) \vert = 0 < \epsilon.
        \]
        
        \item Let \( \epsilon > 0 \) be given. If \( c \) is isolated in \( A \subseteq \mathbb{R} \) then there exists \( \delta > 0 \) s.t. \( V_{\delta}(c) \cap A = \{ c \} \). Thus \( \forall x \in V_{\delta}(c) \cap A \) we have
        \[
        \vert f(x) - f(c) \vert = 0
        \]
        Therefore, for all such \( x \) we have
        \[
        f(x) \in V_{\epsilon}(f(c))
        \]
    \end{enumerate}
    \end{proof}
    
    \item In thoerem 4.3.4, statement (iv) syas that \( f(x)/g(x) \) is continuous at c if both \( f \) and \( g \) are, provided that the quotient is defined. Show that if \( g \) is continuous at \( c \) and \( g(c) \neq 0 \), then there exists an open interval containing \( c \) on which \( f(x)/g(x) \) is always defined.
    
    \begin{proof}
    We begin by noting that \( f(x)/g(x) \) is defined where \( f \) and \( g \) are defined and \( g \) is non-zero. Since, by assumption in Theorem 4.3.4, \( f \) and \( g \) are both defined on the same domain \( A \) we need only then to show that there is an open interval containing \( c \) such that \( g(x) \neq 0 \) for every \( x \) in the interval. To demonstrate, suppose, to the contrary, that every open interval containing \( c \) contains an x such that \( g(x) = 0 \). Then by choosing the sequence of nested intervals given by
    \[
    I_{k} = (c-\frac{1}{k},c+\frac{1}{k})
    \]\
    and selecting a sequence of points
    \[
    x_{k} \in I_{k}
    \]
    such that \( g(x_{k})=0 \), we would get that \( x_{k} \rightarrow c \) while \( g(x_{k}) \rightarrow 0 \) contradciting with \( g(c) \neq 0 \) and g continuous at \( c \). thus there is an open interval, \( I \), containing \( c \) such that \( g(x) \neq 0 \) for every \( x \in I \). Therefore, \( f(x)/g(x) \) is always defined on \( I \).
    \end{proof}
    
    \item \begin{enumerate}
        \item Referring to the proper theorems, give a formal argument that Dirichlet's function from Section 4.1 is nowhere-continuous on \( \mathbb{R} \).
        
        \item Review the definition of Thomae's function in Section 4.1and demonstrate that it fails to be continuous at every rational point.
        
        \item Use the characterization of continuity in theorem 4.3.2 (iii) to show that Thomae's function is continuous at every irrational point in \( \mathbb{R} \). 
    \end{enumerate}
    
    \begin{proof}
    \begin{enumerate}
        \item Let \( c \in \mathbb{R} \). We recall that Direichlet's function is defined by
        \[
        g(x) = \begin{cases}
        1 & x \in \mathbb{Q} \\
        0 & x \not\in \mathbb{Q}
        \end{cases}
        \]
        If \( x_{n} \rightarrow c \) with \( \forall_{n} x_{n} \in \mathbb{Q} \) we get \( g(x) \rightarrow 1 \). On the other hand, if \( y_{n} \rightarrow c \) with \( \forall_{n} y_{n} \not\in \mathbb{Q} \) we get \( g(y_{n}) \rightarrow 0 \). So, by Theorem 4.3.2, it follows that \( g \) is not continuous at \( c \). Therefore, \( g \) is nowhere-continuous on \( \mathbb{R} \).
        
        \item Let \( c \in \mathbb{Q} \). By our Lemma in Exercise 4 from 4.2 we know that if \( x_{n} \rightarrow c \) with \( \forall_{n} x_{n} \in \mathbb{Q} \) we will have \( t(x_{n}) \rightarrow 0 \). However, \( t(c) = 1 \neq 0 \). Thus, again, by Theorem 4.3.2, it follows \( t \) is nowhere-continuous on \( \mathbb{Q} \).
        
        \item Let \( \epsilon > 0 \) be given and let \( c \not\in \mathbb{Q} \). So \( t(c) = 0 \). As was argued in Exercise 4(c) from 4.2 we know that \( S = \{ x \in \mathbb{R} : t(x) \geq \epsilon \} \) has no limit points. Thus \( c \) is not a limit point of \( S \). Therefore there exists \( \delta > 0 \) such that \( x \in V_{\delta}(c) \) implies \( x \not\in S \). Thus, \( x \in V_{\delta}(c) \) implies \( \vert t(x) - t(c) \vert = t(x) < \epsilon \) which is equivalent to  
        \[
        t(V_{\delta}(c)) \subset V_{\epsilon}(t(c))
        \]
        Therefore, \( t \) is continuous at \( c \). 
    \end{enumerate}
    \end{proof}
    
    \item Assume \( h: \mathbb{R} \rightarrow \mathbb{R} \) is continuous on \( \mathbb{R} \) and let \( K = \{ x: h(x) = 0 \} \). Show that \( K \) is a closed set.
    
    \begin{proof}
    Let \( l \) be a limit point of \( K \). So there exists \( k_{n} \rightarrow l \) such that \( \forall_{n} k_{n} \neq l\) and \( k_{n} \in K \). By continuity of \( h \) on \( \mathbb{R} \), we get that 
    \[
    h(k_{n}) \rightarrow h(l)
    \]
    But \( h(k_{n}) = 0 \) for all \( n \). Thus \( h(k_{n}) \rightarrow 0 \). Thus, by uniqueness of the limit, it follows that \( h(l) = 0 \). therefore \( l \in K \). So \( K \) contains all its limit points. Therefore, \( K \) is closed.
    \end{proof}
    
    \item \begin{enumerate}
        \item Show that if a function is continuous on all of \( \mathbb{R} \) and equal to \( 0 \) at every rational point, then it must be identically \( 0 \) on all of \( \mathbb{R} \).
        
        \item If \( f \) and \( g \) are continuous on all of \( \mathbb{R} \) and \( f(r) = g(r) \) at every rational point, must \( f \) and \( g \) be the same function?
    \end{enumerate}
    
    \begin{proof}
    \begin{enumerate}
        \item Suppose, to the contrary, that there exists \( c \not\in \mathbb{Q} \) such that
        \[
        f(c) \neq 0
        \]
        By density of \( \mathbb{Q} \) in \( \mathbb{R} \) there is
        \[
        x_{n} \rightarrow c
        \]
        where \( \forall_{n} x_{n} \neq c \) and \( x_{n} \in \mathbb{Q} \). By continuity of \( f \) at \( c \) we would then have
        \[
        f(x_{n}) \rightarrow f(c)
        \]
        But \( f(x_{n}) \rightarrow 0 \) contradicting with \( f(c) \neq 0 \). Thus \( f(c) = 0 \). Thus \(f(x) = 0 \) for all \( x \in \mathbb{R} \).
        
        \item We claim that, indeed, \( f \) and \( g \) must be the same function. To demonstrate, suppose to the contrary, that there are \( f \) and \( g \), both continuous on \( \mathbb{R} \) and which agree on \( \mathbb{Q} \), which are different. So there exists \( x \not\in \mathbb{Q} \) such that \( f(x) - g(x) \neq 0 \). Thus, if we define \( h(x) = f(x) - g(x) \), by the Algebraic Continuity Theorem, we know that \( h \) is continuous on \( \mathbb{R} \) and that \( h(x) \neq 0 \). By reasoning quite the same as in Exercise 5 above, we know then there is an open interval, I, about \( x \) such that \( h(x) \neq 0 \) for all \( x \in I \). But by density of \( \mathbb{Q} \) in \( \mathbb{R} \), it would then follow that there exists \( r \in \mathbb{Q} \cap I\) such that \( h(r) \neq 0 \). So \( f(r) - g(r) \neq 0 \). So \( f(r) \neq g(r) \) contradicting with our assumption that \( f(r) = g(r) \) for all \( r \in \mathbb{Q} \). 
    \end{enumerate}
    \end{proof}
    
    \item (Contraction Mapping Theorem). Let \( f \) be a function defined on all of \( \mathbb{R} \), and assume there is a constant \( c \) such that \( 0 < c < 1 \) and
    \[
    \vert f(x) - f(y) \vert \leq c \vert x - y \vert
    \]
    for all \( x,y \in \mathbb{R} \).
    \begin{enumerate}
        \item Show that \( f \) is continuous on \( \mathbb{R} \).
        
        \item Pick some point \( y_{1} \in \mathbb{R} \) and construct the sequence
        \[
        (y_{1}, f(y_{1}),f(f(y_{1})),\ldots)
        \]
        In general, if \( y_{n+1} = f(y_{n}) \), show that the resulting sequence \( (y_{n}) \) is a Cauchy sequence. Hence we may let \( y = \lim y_{n} \). 
        
        \item Prove that \( y \) is a fixed point of \( f \).
        
        \item Finally, prove that if \( x \) is any arbitrary point in \( \mathbb{R} \) then the sequence \( (x,f(x),f(f(x)),\ldots) \) converges to \( y \) in (b).
        \end{enumerate}
        
        \begin{proof}
        \begin{enumerate}
            \item Let \( \epsilon > 0 \) be given and let \( z \in \mathbb{R} \). If \( \vert x - z \vert < \frac{\epsilon}{c} \) then
            \[
            \vert f(x) - f(z) \vert < c\vert x-z \vert < c \frac{\epsilon}{c} = \epsilon
            \]
            as desired. 
            
            \item We first wish to show that the sequence \( (y_{n}) \) is bounded. So given \( n \in \mathbb{N} \) we have
            \begin{align*}
                \vert y_{n} \vert &= \vert y_{1} + (y_{2}-y_{1}) + (y_{3}-y_{2}) + (y_{4}-y_{3}) + \ldots + (y_{n-1}+y_{n-2}) + (y_{n}-y_{n-1}) \vert \\
                &\leq \vert y_{1} \vert + \vert y_{2}-y_{1} \vert + \vert y_{3}-y_{2} \vert + \vert y_{4}-y_{3} \vert + \ldots + \vert y_{n}-y_{n-1} \vert \\
                &< \vert y_{1} \vert + \vert y_{2} - y_{1} \vert + c \vert y_{2}-y_{1} \vert + c^{2} \vert y_{2} - y_{1} \vert + \ldots + c^{n-2} \vert y_{2}-y_{1} \vert \\
                \intertext{recognizing that this a geometric series, we get}
                &= \vert y_{1} \vert + \frac{\vert y_{2} - y_{1} \vert(1-c^{n-1})}{1-c} \\
                \intertext{by recalling that \( 0 < c < 1 \) we get that}
                &< \vert y_{1} \vert + \frac{\vert y_{2}-y_{1} \vert}{1-c}
            \end{align*}
            which implies then that \( (y_{n}) \) is bounded. It therefore follows that there exists \( K > 0 \) such that for all \( n \in \mathbb{N} \) we have that \( \vert y_{n} - y_{1} \vert < K \). So, choosing \( N \) so that \( c^{N} < \frac{\epsilon}{K} \) we get that for all \( n > m > N \) we have
            \begin{align*}
                \vert y_{n} - y_{m} \vert &< c \vert y_{n-1} - y_{m-1} \vert \\
                &< c^{2} \vert y_{n-2} - y_{m-2} \vert \\
                &\vdots \\
                &<c^{m-1} \vert y_{n-(m-1)} - y_{1} \vert \\
                &< c^{m-1} K \\
                &< \frac{\epsilon}{K}K \\
                &= \epsilon
            \end{align*}
            Therfore, \( (y_{n}) \) is a Cauchy sequence and we set \( y = \lim y_{n} \). 
            
            \item By (a) we know that \( f \) is continuous. Therefore
            \[
            f(y) = f(\lim y_{n}) = \lim f(y_{n}) = \lim y_{n+1} = y
            \]
            It follows, by a previous theorem, that \( y \) is therefore unique.
            
            \item Choose \( x \in \mathbb{R} \) and let the sequence \( (x_{n}) \) be defined by \( x_{n+1} = f(x_{n}) \). Then choosing \( N \in \mathbb{N} \) so that \( c^{N} < \frac{\epsilon}{\vert y-x \vert} \) implies that for all \( n > N \) we have
            \begin{align*}
                \vert y_{n} - x_{n} \vert &< c \vert y_{n-1} - x_{n-1} \vert \\
                &\vdots \\
                &< c^{n-1} \vert y-x \vert \\
                &< \frac{\epsilon}{\vert y-x \vert} \vert y-x \vert \\
                &= \epsilon
            \end{align*}
            implying that \( \lim x_{n} = \lim y_{n} = y \). 
        \end{enumerate}
        \end{proof}
        
        \item Let \( f \) be a function defined on all of \( \mathbb{R} \) that satisfies the additive condition \( f(x+y) = f(x) + f(y) \) for all \( x,y \in \mathbb{R} \).
        \begin{enumerate}
            \item Show that \( f(0) = 0 \) and that \( f(-x) = -f(x) \) for all \( x \in \mathbb{R} \).
            \item Show that if \( f \) is continuous at \( x=0 \), then \( f \) is continuous at every point in \( \mathbb{R} \).
            \item Let \( k = f(1) \) show that \( f(n) = kn \) for all \( n \in \mathbb{N} \), and then prove that \( f(z) - kz \) for all \( z \in \mathbb{Z} \). Now, prove that \( f(r) = kr \) for all \( r \in \mathbb{Q} \).
            \item Use (b) and (c) to conclude that \( f(x) = kx \) for all \( x \in \mathbb{R} \). Thus, any additive function that is continuous at \( x=0 \) must necessarily be a linear function through the origin.
        \end{enumerate}
        
        \begin{proof}
        \begin{enumerate}
            \item We observe that
            \begin{align*}
                f(0) &= f(0+0) \\
                &= f(0) + f(0) \\
                \intertext{and therefore, by subtracting on both sides by \( f(0) \), we get}
                f(0)-f(0) &= f(0) + f(0) - f(0) \\
                0 &= f(0)
            \end{align*}
            We then have that
            \begin{align*}
                0 &= f(0) \\
                &= f(x+(-x)) \\
                &= f(x) + f(-x) \\
                \intertext{and so, subtracting \( f(x) \) from both sides yields}
                0 -f(x) &= f(x) + f(-x) - f(x) \\
                -f(x) &= f(-x)
            \end{align*}
            
            \item Let \( f \) be continuous at \( 0 \), \(c \in \mathbb{R} \), and \( \epsilon > 0 \) be given. Since \( f \) is continuous at \( 0 \) it follows that there exists \( \delta > 0 \) such that \( \vert y \vert < \delta \) implies \( \vert f(y) \vert < \epsilon \). Thus if \( \vert x-c \vert < \delta \) then we have
            \[
            \vert f(x) - f(c) \vert = \vert f(x-c) \vert < \epsilon
            \]
            So, \( f \) is continuous at \( c \) and so on \( \mathbb{R} \). It follows that \( f \) is continuous on \( \mathbb{R} \). 
            
            \item Let \( n \in \mathbb{N} \) and \( k = f(1) \). We observe that
            \[
            f(n) = f\left(\sum_{i=1}^{n}1\right) = \sum_{i=1}^{n} f(1) = kn
            \]
            If \( z \in \mathbb{Z} \) then either \( z \geq 0 \) or \( z < 0 \). If the former then clearly \( f(z) = kz \). If the latter then \( z = -n \) for some \( n \in \mathbb{N} \).
            \[
            f(z) = f(-n) = -f(n) = -kn = k(-n) = kz
            \]
            Now, let \( r \in \mathbb{Q} \). So \( r \) is of the form \( \frac{p}{q} \) where \( p,q \in \mathbb{Z} \) and \( q \neq 0 \). We first observe that if \( q > 0 \) 
            \begin{align*}
                f(1) &= f\left( \sum_{i=1}^{q} \frac{1}{q} \right) \\
                &= \sum_{i=1}^{q} f\left( \frac{1}{q} \right) \\
                &= f\left( \frac{1}{q} \right)\sum_{i=1}^{q} 1 \\
                &= f\left( \frac{1}{q} \right)q \\
                \intertext{and so}
                \frac{f(1)}{q} &= f\left( \frac{1}{q} \right) \\
                \frac{k}{q} &= f\left( \frac{1}{q} \right) \\
                \intertext{So then}
                -\frac{k}{q} &= -f\left( \frac{1}{q} \right) \\
                 &= f\left( - \frac{1}{q} \right)
            \end{align*}
            and so if \( p = 0 \) we are done. If \( sgn(p) = sgn(q) \) then 
            \begin{align*}
                f\left( \frac{p}{q} \right) &= f\left( \sum_{i=1}^{\vert p \vert} \frac{1}{\vert q \vert} \right) \\
                &= \sum_{i=1}^{\vert p \vert } f\left( \frac{1}{\vert q \vert} \right) \\
                &= f\left( \frac{1}{\vert q \vert} \right) \sum_{i=1}^{\vert p \vert} 1 \\
                &= f\left( \frac{1}{\vert q \vert} \right) \vert p \vert \\
                &= \frac{k}{\vert q \vert} \vert p \vert \\
                &= k \frac{p}{q}
            \end{align*}
            On the other hand, if \( sgn(p) \neq sgn(q) \) then
            \begin{align*}
                f\left( \frac{p}{q} \right) &= f\left( -\frac{\vert p \vert}{\vert q \vert} \right) \\
                &= - f\left( \frac{\vert p \vert}{\vert q \vert} \right) \\
                &= - k \frac{\vert p \vert}{\vert q \vert} \\
                &= k\frac{p}{q}
            \end{align*}
            thus \( f(r) = kr \) for all \( r \in \mathbb{Q} \). 
            
            \item From (b) we know that \( f \) is continuous on all of \( \mathbb{R} \). From (c) we also know that \( f \) and \( g(x) = kx \) must be identical on all of \( \mathbb{Q} \). Therefore, from Exercise 8, we know that \( f=g \) on all of \( \mathbb{R} \). Therefore \( f(x) = kx \) for all \( x \in \mathbb{R} \). 
        \end{enumerate}
        \end{proof}
        
        \item For each of the following choices of \( A \), construct a function \( f: \mathbb{R} \rightarrow \mathbb{R} \) that has discontinuities at every point \( x \) in \( A \) and is continuous on \( A^{c} \).
        \begin{enumerate}
            \item \( A = \mathbb{Z} \)
            \item \( A = \{ x: 0 < x < 1 \} \)
            \item \( A = \{ x: 0 \leq x \leq 1 \} \)
            \item \( A = \{ \frac{1}{n}: n \in \mathbb{N} \} \)
        \end{enumerate}
        \begin{proof}
        \begin{enumerate}
            \item
            \[
            f(x) = \begin{cases} 0 & ,x \in \mathbb{Z} \\ 1 & ,\text{elsewhere} \end{cases}
            \]
            
            \item 
            \[
            f(x) = \begin{cases} 0 & , x \leq 0 \\ x & , x \in \mathbb{Q} \cap (0, \frac{1}{2}] \\ 0 & x \in (\mathbb{R} \setminus \mathbb{Q}) \cap (0,\frac{1}{2}] \\ -x &, x \in \mathbb{Q} \cap (\frac{1}{2}, 1) \\ 0 &, x \in (\mathbb{R} \setminus \mathbb{Q}) \cap (\frac{1}{2}, 1) \\ 0 &, x \geq 1 \end{cases}
            \]
            
            \item 
            \[
            f(x) = \begin{cases} \frac{1}{2} &, x < 0 \\ 0 & x \in (\mathbb{R} \setminus \mathbb{Q}) \cap [0,1] \\ 1 &, x \in \mathbb{Q} \cap [0,1] \\ \frac{1}{2} &, x > 1\end{cases}
            \]
            
            \item 
            \[
            f(x) = \begin{cases} x &, x \in A \\ 2 & x \not\in A \end{cases}
            \]
        \end{enumerate}
        \end{proof}
        
        \item Let \( C \) be the Cantor set constructed in Section 3.1. Define \( g: [0,1] \rightarrow \mathbb{R} \) by
        \[
        g(x) = \begin{cases} 1 & x \in C \\ 0 & x \not\in C \end{cases}
        \]
        \begin{enumerate}
            \item Show that \( g \) fails to be continuous at any point \( c \in C \).
            \item Prove that \( g \) is continuous at every point \( c \not\in C \).
        \end{enumerate}
        \begin{proof}
        \begin{enumerate}
            \item Let \( x \in C \). Recall that \( C \) does not contain any intervals. So \( \forall \delta > 0 \) we have \( V_{\delta}(x) \cap C^{c} \neq \emptyset \). It is easy then to show that \( \exists \epsilon > 0  \: \forall \delta > 0 \: \exists z \in V_{\delta}(x) \) such that \( f(z) \not\in V_{\epsilon}(f(x)) \).
            
            \item Let \( x \not\in C \). 
        \end{enumerate}
        \end{proof}
\end{enumerate}