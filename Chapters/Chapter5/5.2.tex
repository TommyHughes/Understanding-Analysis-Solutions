\section*{5.2}
\begin{enumerate}
    \item Supply the proofs for part (i) and (ii) of Theorem 5.2.4.
    \begin{proof}
    (i) 
    \begin{align*}
        (f+g)'(c) &= \lim_{x \rightarrow c} \frac{(f+g)(x) - (f+g)(c)}{x-c}    \\
        &= \lim_{x \rightarrow c} \frac{f(x)+g(x) - f(c) - g(c)}{x-c} \\
        &= \lim_{x \rightarrow c} \frac{f(x) - f(c)}{x-c} + \frac{g(x) - g(c)}{x-c} \\
        &= \lim_{x \rightarrow c} \frac{f(x) - f(c)}{x-c} + \lim_{x \rightarrow c} \frac{g(x) - g(c)}{x-c} \\
        &= f'(c) + g'(c)
    \end{align*}
    \\
    (ii)
    \begin{align*}
    (kf)'(c) &= \lim_{x \rightarrow c} \frac{(kf)(x) - (kf)(c)}{x-c} \\
    &= \lim_{x \rightarrow c} \frac{kf(x) - kf(c)}{x-c} \\
    &= k \lim_{x \rightarrow c} \frac{f(x) - f(c)}{x-c} \\
    &= k f'(c)
    \end{align*}
    \end{proof}
    
    \item 
    \begin{enumerate}
        \item Use Definition 5.2.1 to produce the proper formula for the derivative of \( f(x) = \frac{1}{x} \). 
        \item Combine the result in (a) with the chain rule (Theorem 5.2.5) to supply a proof for part (iv) of Theorem 5.2.4.
        \item Supply a direct proof of Theorem 5.2.4 (iv) by algebraically manipulating the difference quotient for \( \frac{f}{g} \) in a style similar to the proof of Theorem 5.2.4 (iii). 
    \end{enumerate}
    
    \begin{proof}
    \begin{enumerate}
        \item Let \( c \in \mathbb{R} \) such that \( c \neq 0 \). Then
        \begin{align*}
            \lim_{x \rightarrow c} \frac{\frac{1}{x}- \frac{1}{c}}{x-c} &= \lim_{x \rightarrow c} \frac{\frac{c}{cx} - \frac{x}{cx}}{x-c} \\
            &= \lim_{x \rightarrow c} \frac{-(x-c)}{(cx)(x-c)} \\
            &= \lim_{x \rightarrow c} -\frac{1}{cx} \\
            &= - \frac{1}{c^{2}}
        \end{align*}
        Thus, in general, \( f'(x) = -\frac{1}{x^{2}} \)
        
        \item Notice we can rewrite \( \frac{f}{g} \) as \( fg^{-1} \). Thus, if \( h(x) = \frac{1}{x} \) then \( (h \circ g) = \frac{1}{g} = g^{-1}\) which, in turn implies \( \frac{f}{g} = fg^{-1} = f(h \circ g) \). Thus, by (iii) of Theorem 5.2.4, we get that when \( c \) is chosen so that \( g(c) \neq 0 \) we have
        \[
        \left(\frac{f}{g}\right)'(c) = (f(h \circ g))'(c) =  f'(h \circ g)(c) + f (h \circ g)'(c)
        \]
        By The Chain Rule, we get that \( (h \circ g)'(c) = h'(g(c))g'(c) \). From (a), we know that
        \[
        h'(g(c))g'(c) = -\frac{1}{(g(c))^{2}}g'(c)
        \]
        Thus, we have
        \begin{align*}
        f'(h \circ g)(c) + f (h \circ g)'(c) &=  \frac{f'(c)}{g(c)}  + f(c)\left( -\frac{1}{(g(c))^{2}}g'(c) \right) \\
        &=  \frac{f'(c)}{g(c)} + \left( -\frac{f(c)g'(c)}{(g(c))^{2}} \right) \\
        &=  \frac{g(c)f'(c) - f(c)g'(c)}{(g(c))^{2}}
        \end{align*}
        So, altogether, we get that
        \[
        \left(\frac{f}{g}\right)'(c) =  \frac{g(c)f'(c) - f(c)g'(c)}{(g(c))^{2}} 
        \]
        as desired.
        
        \item Let \( c \) be chosen such that \( g(c) \neq 0 \). Then
        \begin{align*} 
            \left( \frac{f}{g} \right)'(c) &= \lim_{x \rightarrow c} \frac{\frac{f(x)}{g(x)}- \frac{f(c)}{g(c)}}{x-c} \\
            &= \lim_{x \rightarrow c} \frac{\frac{g(c)f(x) - f(c)g(x)}{g(c)g(x)}}{x-c} \\
            &= \lim_{x \rightarrow c} \frac{1}{x-c}\left( \frac{g(c)f(x) + f(x)g(x) - f(x)g(x) - f(c)g(x)}{g(x)g(c)} \right) \\
            &= \lim_{x \rightarrow c} \frac{1}{x-c} \left( \frac{f(x)(g(c)-g(x)) + g(x)(f(x) - f(c))}{g(x)g(c)} \right) \\
            &= \lim_{x \rightarrow c} \frac{1}{x-c} \left( \frac{ g(x)(f(x) - f(c))-f(x)(g(x)-g(c))}{g(x)g(c)} \right) \\
            &= \lim_{x \rightarrow c} \frac{1}{g(c)} \left( \frac{f(x)-f(c)}{x-c} - \frac{f(x)}{g(x)}\frac{g(x)-g(c)}{x-c} \right) \\
            \intertext{Now, since \( f \) and \( g \) are differentiable at \( c \) and \( g(c) \neq 0 \), it follows that both \( f \) and \( g \) are continuous at \( c \). So we have}
            &= \frac{1}{g(c)}\lim_{x \rightarrow c} \frac{f(x)-f(c)}{x-c} - \left( \lim_{x \rightarrow c} \frac{f(x)}{g(x)} \right) \left(\lim_{x \rightarrow c} \frac{g(x)-g(c)}{x-c}\right) \\
            &= \frac{1}{g(c)} \left( f'(c) - \frac{f(c)}{g(c)}g'(c) \right) \\
            &= \frac{f'(c)}{g(c)} - \frac{f(c)g'(c)}{g(c)^{2}} \\
            &= \frac{g(c)f'(c) - f(c)g'(c)}{g(c)^{2}}
        \end{align*}
        as desired.
    \end{enumerate}
    \end{proof}
    
    \item By imitating the Dirichlet constructions in Section 4.1, construct a function on \( \mathbb{R} \) that is differentiable at a single point.
    \begin{proof}
    Let
    \[
    f(x) = \begin{cases} x^{2} & x \in \mathbb{Q} \\ 0 & x \not\in \mathbb{Q} \end{cases}
    \]
    First, to demonstrate that \( f \) is differentiable at \( x= 0 \), we let \( \epsilon > 0 \) be given. If \( \vert x \vert < \epsilon \), then, for all such \( x \) we have either
    \[
    f(x) = 0
    \]
    or
    \[
    f(x) = x^{2}
    \]
    If \( f(x) = 0 \) then
    \[
    \left| \frac{f(x)-f(0)}{x-0} \right| = \left| \frac{0-0}{x-0} \right| = 0 < \epsilon
    \]
    On the other hand, if \( f(x) = x^{2} \) then
    \[
    \left| \frac{f(x)-f(0)}{x-0} \right| = \left| \frac{x^{2}-0}{x-0} \right| = \left| x \right| < \epsilon 
    \]
    Thus
    \[
    \lim_{x \rightarrow 0 } \frac{f(x) - f(0)}{x-0} = 0
    \]
    Now, to show that \( f \) is not differentiable at any other point, let \( c \neq 0 \). Now, there exist sequences \( (x_{n}) \subset \mathbb{Q} \) and \( (y_{n}) \subset \mathbb{R}\setminus\mathbb{Q} \) such that \( x_{n}, y_{n} \rightarrow c \) while, for every \( n \), \( x_{n} \neq c \) and \( y_{n} \neq c \). If \( f(c) = 0 \) then
    \begin{align*}
        \frac{f(x_{n})-f(c)}{x_{n}-c} &= \frac{x_{n}^{2}}{x_{n}-c} \\
        &= \frac{x_{n}^{2} -c^{2}+c^{2}}{x_{n}-c} \\
        &= \frac{(x_{n}+c)(x_{n}-c)+c^{2}}{x_{n}-c} \\
        &= x_{n}+c + \frac{c^{2}}{x_{n}-c} \\
    \end{align*}
    and clearly
    \[
    x_{n}+c + \frac{c^{2}}{x_{n}-c} \rightarrow \pm\infty
    \]
    Thus \( \lim_{x \rightarrow c} \frac{f(x) - f(c)}{x-0} \) does not exist. On the other hand, if \( f(c) = c^{2} \) then 
    \begin{align*}
        \frac{f(y_{n})-f(c)}{y_{n}-c} &= \frac{0-c^{2}}{y_{n}-c} \\
        &= - \frac{c^{2}}{y_{n}-c}
    \end{align*}
    Again, clearly we have
    \[
    -\frac{c^{2}}{y_{n}-c} \rightarrow \pm\infty
    \]
    Thus, \( \lim_{x \rightarrow c} \frac{f(x)-f(c)}{x-c} \) does not exist when \( f(c) = c^{2} \). Thus, in all cases, when \( c \neq 0 \) we have that \( \lim_{x \rightarrow c} \frac{f(x)-f(c)}{x-c} \) does not exist. Therefore, \( f \) is differentiable    a solely at \( c = 0 \). 
    \end{proof}
    
    \item Let \( f_{a}(x) = \begin{cases} x^{a} & x \geq 0 \\ 0 & x < 0 \end{cases} \)
    \begin{enumerate}
        \item For which values of \( a \) is \( f \) continuous at zero?
        \item For which values of \( a \) is \( f \) differentiable at zero? In this case, is the derivative function continuous?
        \item For which values of \( a \) is \( f \) twice-differentiable?
    \end{enumerate}
    First a few Lemmas. Throughout, we assume some basic properties of \( \ln(x) \) and \( e^x \) (such as monotonicity) without proof.
    \begin{lemma}
     For \( b > 0 \) and \( b \neq 1 \), \( b^{x} \) is injective.
    \end{lemma}
    
    \begin{proof}
    Let \( b>0 \) and \( b \neq 1 \). Now suppose
    \begin{align*}
        b^{x_{1}} &= b^{x_{2}} \\
        \intertext{Wlog, assume that \( x_{1} \geq x_{2} \), so we have}
        b^{x_{2}+k} &= b^{x_{2}} \\
        b^{x_{2}}b^k &= b^{x_{2}} \\
        b^k &= 1 \\
    \end{align*}
    We claim this then implies that \( k=0 \). For otherwise we would have that
    \[
    b = \left( b^k \right)^{\frac{1}{k}} = 1^\frac{1}{k} = 1
    \]
    contradicting with our assumption that \( b \neq 1 \). So \( k = 0 \). Therefore, \( x_{1} = x_{2} \).
    \end{proof}
    
    \begin{corollary}
    \( log_{b}(x) \) is injective.
    \end{corollary}
    
    \begin{lemma}
    \( \ln(x) \) is continuous at \( x =1 \).
    \end{lemma}
    
    \begin{proof}
    Let \( \epsilon > 0 \) be given and choose \( \delta = \min\{ \vert e^{-\epsilon} -1 \vert, e^\epsilon - 1 \} \). Then \( \vert x-1 \vert < \delta \) implies that
    \begin{center}
    \begin{tabular}{rcccl}
    \( e^{-\epsilon}-1 \) & \( < \) & \( x-1 \) & \( < \) & \( e^\epsilon -1 \) \\
    \( e^{-\epsilon} \) & \( < \) & \( x \) & \( < \) & \( e^\epsilon \) \\
    \( -\epsilon \) & \( < \) & \( \ln(x) \) & \( < \) & \( \epsilon \) \\
    \end{tabular}
    \end{center}
    This implies that \( \vert \ln(x) \vert < \epsilon \) which, in turn, implies that
    \[
    \vert \ln(x) - \ln (1) \vert = \left| \ln\left(\frac{x}{1}\right) \right| = \vert \ln(x) \vert < \epsilon
    \]
    Therefore, \( \ln(x) \) is continuous at \( x = 1 \).
    \end{proof}
    
    \begin{corollary}
    \( \ln(x) \) is continuous on \( (0,\infty) \).
    \end{corollary}
    \begin{proof}
    Let \( \epsilon > 0 \) be given and \( c \in (0,\infty) \). So, we have
    \begin{align*}
        \vert \ln(x) - \ln(c) \vert &= \left| \ln\left( \frac{x}{c} \right) \right|
        \intertext{and so if \( z = \frac{x}{c} \) then we have }
        &= \vert \lim(z) \vert
    \end{align*}
    Now, by our previous Lemma, we know there exists \( \delta > 0 \) such that \( \vert z-1 \vert < \delta \) implies \( \vert \ln(z) \vert < \epsilon \). So then \( \left| \frac{x}{c} - 1 \right| < \delta \) implies that \( \left| \ln\left( \frac{x}{c} \right) \right| < \epsilon \). Thus \( \vert x-c \vert < c\delta \) implies
    \[
    \left| \ln(x) - \ln(c) \right| = \left| \ln\left(\frac{x}{c}\right) \right| < \epsilon
    \]
    Therefore, \( \ln(x) \) is continuous on \( (0,\infty) \). 
    \end{proof}
    
    \begin{lemma} If \( c \in (0,\infty) \) then 
    \( \lim_{x\rightarrow c} \ln(x^{a}) = \ln(c^{a}) \)
    \end{lemma}
    
    \begin{proof}
    \begin{align*}
        \lim_{x\rightarrow c} \ln(x^{a}) &= \lim_{x\rightarrow c} a\ln(x) \\
        &= a \lim_{x\rightarrow c} \ln(x) \\
        \intertext{and by continuity, we have}
        &= a \ln(c) \\
        &= \ln(c^{a})
    \end{align*}
    \end{proof}
    
    \begin{corollary} \( x^a \) is continuous on \( (0, \infty) \). 
    \end{corollary}
    
    \begin{proof}
    Let \( c \in (0,\infty) \). By injectivity and continuity of \( \ln(x) \) and the above Lemma, we have
    \[
    \ln(\lim_{x\rightarrow c} x^a) = \lim_{x \rightarrow c} \ln(x^a) = \ln(c^a) \iff \lim_{x\rightarrow c} x^a = c^a
    \]
    \end{proof}
    
    \begin{lemma}
    \( \frac{d}{dx}[\ln(x)] = \frac{1}{x} \) for \( x \in (0, \infty) \)
    \end{lemma}
    
    \begin{proof}
    Let \( c \in (0,\infty) \). We simply compute
    \begin{align*}
        \lim_{x \rightarrow c} \frac{\ln(x)-\ln(c)}{x-c} &= \lim_{x \rightarrow c} \frac{\ln\left( \frac{x}{c} \right)}{x-c} \\
        &= \lim_{x \rightarrow c} \frac{\ln\left( 1 + \frac{x-c}{c} \right)}{x-c} \\
        &= \lim_{x \rightarrow c} \ln\left[ \left( 1+ \frac{x-c}{c} \right)^{\frac{1}{x-c}} \right] \\
        &= \lim_{x \rightarrow c} \ln\left[ \left( 1 + \frac{\frac{1}{c}}{\frac{1}{x-c}} \right)^{\frac{1}{x-c}} \right] \\
        \intertext{and by continuity of \( \ln(x) \), we have}
        &=\ln\left[ \lim_{x \rightarrow c} \left( 1 + \frac{\frac{1}{c}}{\frac{1}{x-c}} \right)^{\frac{1}{x-c}} \right] \\
        \intertext{and by definition of \( e^x \), we have}
        &= \ln(e^{\frac{1}{c}}) \\
        &= \frac{1}{c}
    \end{align*}
    \end{proof}
    
    \begin{corollary}
    \( \frac{d}{dx}[x^a] = ax^{a-1} \) for \( a \in \mathbb{R} \) and \( x \in (0, \infty) \).
    \end{corollary}
    
    \begin{proof}
    Let \( a \in \mathbb{R} \) and \( x \in (0, \infty) \). Then by the above Lemma, the Chain Rule, and linearity of the derivative, we have the following
    \[
    \frac{1}{x^a} \frac{d}{dx}[x^a] = \frac{d}{dx}[\ln(x^a)] = \frac{d}{dx}[a\ln(x)] = a \frac{d}{dx}[\ln(x)] = a \frac{1}{x}
    \]
    Thus, 
    \begin{align*}
        \frac{1}{x^a} \frac{d}{dx}[x^a] &= a \frac{1}{x} \\
        \intertext{which yields}
        \frac{d}{dx}[x^a] &= ax^{a-1}
    \end{align*}
    as desired.
    \end{proof}

    Now we continue with our proof
    \begin{proof}
    \begin{enumerate}
        \item We claim \( f_a \) is continuous at zero iff \( a > 0 \). To begin, we observe that \( f_a(0) = 0^a \) is defined only when \( a > 0 \). Thus if \( f_a \) is continuous at zero then \( a > 0 \). Now if \( a > 0 \) and \( \epsilon > 0 \) is given, \( \vert x \vert < \sqrt[a]{\epsilon} \) implies
        \[
        \vert f_a(x) - f_a(0) \vert = \vert 0 - 0 \vert = 0 < \epsilon
        \]
        whenever \( -\sqrt[a]{\epsilon} < x < 0 \) and
        \[
        \vert f_a(x) - f_a(0) \vert = \vert x^a - 0 \vert = \vert x^a \vert = \vert x \vert^a < \sqrt[a]{\epsilon}^a = \epsilon
        \]
        whenever \( 0 < x < \sqrt[a]{\epsilon} \). Therefore if \( a > 0 \) then \( f_a \) is continuous at zero.
        
        \item Notice that \( f_a \) is differentiable at zero iff
        \[
        \lim_{x \rightarrow 0^-} f_a(x) = 0 = \lim_{x \rightarrow 0^+} f_a(x)
        \]
        We claim that \( 0 = \lim_{x \rightarrow 0^+} f_a(x) \) iff \( a > 1 \). So if \( 0 = \lim_{x \rightarrow 0^+} f_a(x) \), then for every \( \epsilon > 0\) there exists \( \delta > 0 \) such that \( 0 < x < \delta \) implies
        \[
        \left| \frac{f_a(x)-f_a(0)}{x} \right| = \left| \frac{x^a}{x} \right| = \vert x^{a-1} \vert = x^{a-1} < \epsilon
        \]
        However, if \( a =1 \) then \( x^{a-1} = 1 \) for all \( x \) such that \( 0 < x < \delta \). So for \( \epsilon = \frac{1}{2} \) we would then have that \( 1 < \frac{1}{2} \) which is clearly absurd. If \( a < 1 \) then \( 0 < 1-a \) and we can therefore choose \( x \) so that \[
        0 < x^{1-a} < \min\{ \delta, 1/\epsilon \}
        \] 
        yielding \( x^{a-1} = \frac{1}{x^{1-a}} > \frac{1}{1/\epsilon} = \epsilon \) which contradicts with our assumption that \( 0 < z < \delta \) implies \( z^{a-1} < \epsilon \). Therefore, \( a > 1 \).
        
        \vspace{3 mm}
        
        Now, if \( a > 1 \) then \( 0 < x < \epsilon^{\frac{1}{a-1}} \) implies
        \[
        \left| \frac{f_a(x)- f_a(0)}{x-0} \right| = x^{a-1} < (\epsilon^{\frac{1}{a-1}})^{a-1} = \epsilon
        \]
        and, therefore, \( 0 = \lim_{x \rightarrow 0^+} f_a(x) \). Therefore \( f_a \) is differentiable at zero iff \( a > 1 \). 
        
        \vspace{3mm}
        
        Now we wish to show that, in this case, indeed, \( f_a' \) is continuous. It is obvious from the definition of \( f_a \) and our previous argument that \( f_a' \) is continuous on \( (\infty, 0] \). By our Corollary we know that, for \( x \in (0, \infty) \) \( \frac{d}{dx}[x^a] = ax^{a-1} \). By our Lemma  we know that, for \( a \in \mathbb{R} \), \( x^a \) is continuous on \( (0, \infty) \). In particular, for \( x^{a-1} \). Thus, \(f_a' \) is continuous on \( (0,\infty) \). Therefore, \( f_a' \) is continuous when \( a > 1 \). 
        
        \item Notice, from previous considerations, we get that \( f_a \) is infinitely differentiable on \( (-\infty, 0) \cup (0, \infty) \). However, for \( f_a \) to be twice-differentiable on \( \mathbb{R} \) we must have
        \[
        \lim_{x \rightarrow 0^-} \frac{f_a'(x)-f_a'(0)}{x-0} = 0 = \lim_{x \rightarrow 0^+} \frac{f_a'(x)-f_a'(0)}{x-0} 
        \]
        But observe that
        \begin{align*}
            \lim_{x \rightarrow 0^+} \frac{f_a'(x)-f_a'(0)}{x-0} &= \lim_{x \rightarrow 0^+} \frac{f_a'-0}{x-0} \\
            &= \lim_{x \rightarrow 0^+} \frac{ax^{a-1}}{x} \\
            &= \lim_{x \rightarrow 0^+} ax^{a-2}    
        \end{align*}
        and \( \lim_{x \rightarrow 0^+} ax^{a-2} = 0 \) if and only if \( a > 2 \). 
    \end{enumerate}
    \end{proof}
    
    \item Let
    \[
    g_{a}(x) = \begin{cases} x^{a}\sin\left( \frac{1}{x} \right)  & x \neq 0 \\ 0 & x = 0 \end{cases}
    \]
    Find a particular (potentially noninteger) value for \( a \) so that
    \begin{enumerate}
        \item \( g_{a} \) is differentiable on \( \mathbb{R} \) but such that \( g_{a}' \) is unbounded on \( [0,1] \).
        
        \item \( g_{a} \) is differentiable on \( \mathbb{R} \) with \( g_{a}' \) continuous but not differentiable at zero.
        
        \item \( g_{a} \) is differentiable on \( \mathbb{R} \) and \( g_{a}' \) is differentiable on \( \mathbb{R} \), but such that \( g_{a}'' \) is not continuous at zero. 
    \end{enumerate}
    
    \begin{proof}
    \begin{enumerate}
        \item 
    \end{enumerate}
    \end{proof}
    
    \item
    \begin{enumerate}
        \item Assume that \( g \) is differentiable on \( [a,b] \) and satisfies \( g'(a) < 0 < g'(b) \). Show that there exists a point \( x \in (a,b) \) where \( g(a) > g(x) \), and a point \( y \in (a,b) \) where \( g(y) < g(b) \)
        
        \item Now complete the proof of Darboux's Theorem
    \end{enumerate}
    
    \begin{lemma}
    Let \( f \) be differentiable on \( [a,b] \). If \( c \in [a,b] \) is a minimum for \( f \) on \( [a,b] \), then \( c \) is either an endpoint of \( [a,b] \) or \( f'(c) = 0 \)
    \end{lemma}
    
    \begin{proof}
    Let the assumptions on \( f \) and \( c \) hold, and suppose that \( c \) is not an endpoint of \( [a,b] \). By differentiability of \( f \) at \( c \), we have
    \[
    \lim_{x \rightarrow c^-} \frac{f(x)-f(c)}{x-c} = \lim_{x \rightarrow c^+} \frac{f(x)-f(c)}{x-c}
    \]
    By our assumptions on \(c\), we have \( f(x) -f(c) \geq 0 \) for all \( x \in [a,b] \). Thus we have
    \[
    \lim_{x \rightarrow c^-} \frac{f(x)-f(c)}{x-c} \leq 0 \leq \lim_{x \rightarrow c^+} \frac{f(x)-f(c)}{x-c}    
    \]
    Together, with the above equality, we get then that
    \[
    f'(c) = \lim_{x \rightarrow c} \frac{f(x)-f(c)}{x-c} = 0
    \]
    \end{proof}
    
    We now proceed with the exercise.
    \begin{proof}
    
    \begin{enumerate}
        \item We have
        \begin{align*}
            \lim_{x \rightarrow a^+} \frac{g(x) - g(a)}{x-a} &< 0 \\
            \intertext{and}
            \lim_{y \rightarrow b^-} \frac{g(y)-g(b)}{y-b} &> 0 
        \end{align*}
        Since \( x > a \) and \( y<b \), then the respective inequalities, along with the Sequential Criterion for Functional Limits and the Order Limit Theorem, imply that there exist \( x,y \in (a,b) \) such that
        \begin{align*}
            g(x) &< g(a) \\
            g(y) &< g(b)
        \end{align*}
        
        \begin{corollary}
        The assumptions in (a) imply that the minimum of \( g \) on \( [a,b] \) is not at the end points.
        \end{corollary}
        
        \begin{proof}
        From (a), we have
        \[
        \min\{ g(x), g(y) \} < g(a), g(b)
        \]
        \end{proof}
        
        \item To demonstrate Darboux's Theorem, we let \( f \) be differentiable on \( [a,b] \) and let \( \alpha \) be such that \( f'(a) < \alpha < f'(b) \) or \( f'(b) < \alpha < f'(a) \). If we define \( g(x) = f(x) - \alpha x \) we get that \( g'(x) = f'(x) - \alpha \). By differentiability of \( g \) on \( [a,b] \), it follows that \( g \) is continuous on \( [a,b] \) and so, by compactness, \( g \) achieves a minimum, \( c \), on \( [a,b] \). Now, from the above inequalities, we get that
        \[
        g'(a) < 0 < g'(b)
        \]
        or
        \[
        g'(b) < 0 < g'(a)
        \]
        and therefore, by (a), that \(c \) is not an endpoint of \( [a,b] \). Thus \( c \in (a,b) \). Furthermore, from the Lemma, we get that we must then have that \( g'(c) = 0 \). Thus \( f'(c) - \alpha = 0 \) which is equivalent to \( f'(c) = \alpha \). Therefore, there exists \( c \in (a,b) \) such that \( f'(c) = \alpha \). 
    \end{enumerate}
    
    \end{proof}
    
    
    
    
    
    
    
    
    
    
\end{enumerate}