\section*{6.3}
\begin{enumerate}
    \item 
    \begin{enumerate}
        \item Let
        \[
        h_n(x) = \frac{\sin(nx)}{n}
        \]
        Show that \( h_n \rightarrow 0 \) uniformly on \( \mathbb{R} \). At what points does the sequence of derivatives \( h'_n \) converge?
        
        \item Modify this example to show that it is possible for a sequence \( (f_n) \) to converge uniformly but for \( (f'_n) \) to be unbounded.
    \end{enumerate}
    
    \begin{proof}
    \begin{enumerate}
        \item Let \( \epsilon > 0 \) be given. We observe then that if \( \frac{1}{\epsilon} < N \) then \( n \geq N \) implies that
        \[
        -\epsilon < -\frac{1}{n} \leq \frac{\sin(nx)}{n} \leq \frac{1}{n} < \epsilon
        \]
        and therefore, that
        \[
        \left| \frac{\sin(nx)}{n} \right| < \epsilon 
        \]
        for all \( x \). 
        
        \item It is quite easy to see that if \( g_n = \frac{\sin(n^2x)}{n} \), then \( g_n \rightarrow 0 \) uniformly. On the other hand
        \[
        g_n' = n \cos(n^2x)
        \]
        and therefore \( (g_n') \) is unbounded. 
    \end{enumerate}
    \end{proof}
    
    \item Consider the sequence of functions defined by
    \[
    g_n(x) = \frac{x^n}{n}
    \]
    \begin{enumerate}
        \item Show \( (g_n) \) converges uniformly on \( [0,1] \) and find \( g = \lim g_n \). Show that \( g \) is differentiable and compute \( g'(x) \) for all \( x \in [0,1] \).
        
        \item Now, show that \( (g'_n) \) converges on \( [0,1] \). Is the convergence uniform? Set \( h = \lim g'_n \) and compare \( h \) and \( g' \). Are they the same? 
    \end{enumerate}
    
    \begin{proof}
    \begin{enumerate}
        \item We claim \( g_n \rightarrow 0 \) uniformly on \( [0,1] \). To demonstrate, let \( \epsilon > 0 \) be given. If \( N > \frac{1}{\epsilon} \), then for all \( n \geq N \) and for all \( x \in [0,1] \), we have
        \[
        \left| \frac{x^n}{n} \right| = \frac{x^n}{n} \leq \frac{1}{n} < \epsilon
        \]
        therefore, \( g_n \rightarrow 0 \) on \( [0,1] \). Clearly \( g(x) = 0 \) is differentiable and \( g'(x) = 0 \) for all \( x \in [0,1] \).
        
        \item First, it is easy to see that
        \[
        g_n'(x) = x^n-1
        \]
        So, for \( x \in [0,1) \), we have
        \[
        \lim_{n \rightarrow \infty} x^{n-1} = 0
        \]
        and for \( x =1 \), we have
        \[
        \lim_{n \rightarrow \infty} x^{n-1} = 1
        \]
        Thus, if
        \[
        h(x) = \begin{cases} 0 & 0 \leq x < 1 \\ 1 & x = 1\end{cases}
        \]
        the \( g_n' \rightarrow h \) on \( [0,1] \). We see that \( (g_n') \) does not converge uniformly on \( [0,1] \) since each \( (g_n') \) is continuous at \( 1 \) while \( h \) is not. Furthermore, we see that \( h \neq g' \) since \( h(1) = 1 \) and \( g'(1) = 0 \). 
    \end{enumerate}
    \end{proof}
    
    \item Consider the sequence of functions
    \[
    f_n(x) = \frac{x}{1+nx^2}
    \]
    It can be shown from a previous exercise (6.2.4) that \( (f_n) \) converges uniformly on \( \mathbb{R} \). Now, let \( f = \lim f_n \). Compute \( f_n'(x) \) and find all the values of \( x \) for which \( f'(x) = \lim f_n'(x) \). 
    \begin{proof}
    We know from Exercise 6.2.4 that \( f_n \rightarrow 0 \) uniformly on \( \mathbb{R} \). Thus \( f'(x) = 0 \). On the other hand
    \[
    f_n'(x) = \frac{1-nx^2}{(1+nx^2)^2}
    \]
    and thus, it is a simple matter to show that, for \( x \neq 0 \) we get that \( \lim f_n'(x) = 0 \). On the other hand, if \( x = 0 \), then \( \lim f_n'(0) = 1 \). Therefore, \( \lim f_n'(x) = f'(x) \) for all \( x \neq 0 \). 
    \end{proof}
    
    \item Let 
    \[
    g_n(x) = \frac{nx+x^2}{2n}
    \]
    and set \( g(x) = \lim g_n(x) \). Show that \( g \) is differentiable in two ways:
    \begin{enumerate}
        \item Compute \( g(x) \) by algebraically taking the limit as \( n \rightarrow \infty \) and then find \( g'(x) \).
        
        \item Compute \( g_n'(x) \) for each \( n \in \mathbb{N} \) and show that the sequence of derivatives \( (g_n') \) converges uniformly on every interval \( [-M,M] \). Use Theorem 6.3.3 to conclude \( g'(x) = \lim g_n'(x) \).
    \end{enumerate}
    
    \begin{proof}
    \begin{enumerate}
        \item \[
        \lim g_n(x) = \lim \frac{nx + x^2}{2n} = \lim \frac{x + \frac{x^2}{n}}{2} = \frac{x}{2} = g(x)
        \]
        Therefore,
        \[
        g'(x) = \frac{1}{2}
        \]
        
        \item We first compute
        \[
        g_n'(x) = \frac{1}{2} + \frac{x}{n}
        \]
        We claim that \( g_n' \rightarrow \frac{1}{2} \) uniformly on \( [-M,M] \) for all \( M > 0 \). To this end, we let \( \epsilon > 0 \) be given. If \( \frac{M}{\epsilon} < N \) then we will have, for all \( n \geq N \) and all \( x \in [-M, M ] \)
        \[
        \left| \frac{1}{2} + \frac{x}{2} - \frac{1}{2} \right| =  \ \frac{\vert x \vert}{n} < \frac{M}{n} < \frac{M}{\frac{M}{\epsilon}} = \epsilon
        \]
        Therefore \( g_n' \rightarrow \frac{1}{2} \) uniformly on \( [-M,M] \) for all \( M > 0 \). Notice then that, for \( x= 0 \) we have that \( g_n(0) \rightarrow 0 \). Since \( 0 \in [-M,M] \) for all \( M > 0 \), it follows that \( g_n \rightarrow g \) uniformly on \( [-M, M ] \) for all \( M > 0 \). That is, \( g_n \rightarrow g \) uniformly on \( \mathbb{R} \). 
    \end{enumerate}
    \end{proof}
    
    \item Prove Theorem 6.3.2
    
    \begin{proof}
    Let \( \epsilon > 0 \) be given. So we have
    \[
    \vert f_n(x) - f_m(x) \vert \leq \vert f_n(x) - f_m(x) - (f_n(x_0) - f_m(x_0)) \vert + \vert f_m(x_0) - f_n(x_0) \vert
    \]
    So, applying the Mean Value Theorem to \( f_n-f_m \) yields
    \[
    \vert f_n(x) - f_m(x) \vert \leq \vert f_n'(c) - f_m'(c) \vert \vert x - x_0 \vert + \vert f_m(x_0) - f_n(x_0) \vert \leq \vert f_n'(c) - f_m'(c) \vert (b-a) + \vert f_m(x_0) - f_n(x_0) \vert
    \]
    for some \( c \) between \( x \) and \( x_0 \). 
    Thus by convergence of \( (f_n(x_0)) \) and by uniform convergence of \( (f_n') \), we get that there exists \( N \in \mathbb{N} \) such that, for all \( n \geq N \) and for all \( x \in [a,b] \) we have
    \[
    \vert f_n(x) - f_m(x) \vert \leq \vert f_n'(c) - f_m'(c) \vert (b-a) + \vert f_m(x_0) - f_n(x_0) \vert < \frac{\epsilon}{2(b-a)} (b-a) + \frac{\epsilon}{2} = \epsilon
    \]
    Therefore, by the Cauchy Criterion, we have that \( (f_n) \) is uniformly convergent on \( [a,b] \). 
    \end{proof}
\end{enumerate}